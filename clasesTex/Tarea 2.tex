\documentclass[12pt]{article}
 
\usepackage[english]{babel}
\usepackage[utf8]{inputenc}
\usepackage{amsmath}
\usepackage{graphicx}
\usepackage{parskip}

\topmargin 0.0cm
\oddsidemargin 0.2cm
\textwidth 16cm 
\textheight 22cm
\footskip 1.0cm

\usepackage{fancyhdr}
\pagestyle{fancy}
\lhead{\begin{picture}(-15,-15) \put(-15,-15){\includegraphics[width=5cm]{Logo.jpg}} \end{picture}}

\begin{document}
\textbf{Econometría de Series de Tiempo}\\
\textbf{Profesor: Marcelo Villena, PhD }\\ 

\begin{center}
\textbf{{\large Tarea 1}}\\
Fecha entrega: $2-ago-2018$ \\ 
\end{center}

\textbf{Parte I: Te\'orica}
\vspace{5mm}
\begin{enumerate}
\item Encontrar las distintas medidas de dependencia revisadas en clases, para los modelos: random walk, autorregresivo y de media m\'ovil.\\
\end{enumerate}

\textbf{Parte II: Pr\'actica} 
\vspace{5mm}\\
Escoger 3 activos de la misma industria y realizar los siguientes an\'alisis:
\begin{enumerate}
\item Comprobar que todos los activos tienen un $\beta$ simular.
\item Analizar si el performance de la regresi\'on mejora o empeora con datos de distinta frecuencia (Ej: diaria, mensual, trimestral).
\item Usando el software R, calcule matricialmente la f\'ormula de m\'\i{}nimos cuadrados ordinarios y compruebe los resultados de una de sus regresiones.
\item ?`Cambi\'o estructuralmente el $\beta$ de la industria despu\'es de la crisis financiera?
\item ?`Son sus activos elegidos random walk? ?`cu\'al es la implicancia para la hip\'otesis de mercado eficiente comentada en clases?

\end{enumerate}


\end{document}
